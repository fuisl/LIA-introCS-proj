\documentclass[10pt,twocolumn,letterpaper]{article}
\usepackage{array}
\usepackage{statcourse}
\usepackage{times}
\usepackage{epsfig}
\usepackage{graphicx}
\usepackage{amsmath}
\usepackage{amssymb}
\usepackage{booktabs}
\usepackage{multirow}
% Include other packages here, before hyperref.

% If you comment hyperref and then uncomment it, you should delete
% egpaper.aux before re-running latex.  (Or just hit 'q' on the first latex
% run, let it finish, and you should be clear).
\usepackage[breaklinks=true,bookmarks=false]{hyperref}


\statcoursefinalcopy


\setcounter{page}{1}
\begin{document}


%%%%%%%%%%%%%%%%%%%%%%%%%%%%%%%%%%%%%%%%%%%%%%%%%%%%%%%%%%%%%%%
% DO NOT EDIT ANYTHING ABOVE THIS LINE
% EXCEPT IF YOU LIKE TO USE ADDITIONAL PACKAGES
%%%%%%%%%%%%%%%%%%%%%%%%%%%%%%%%%%%%%%%%%%%%%%%%%%%%%%%%%%%%%%%



%%%%%%%%% TITLE
\title{Comparative Analysis of Machine Learning Models: \\Alexnet, VGG, Resnet, YOLO}

\author{Pham Duc An\\
{\tt\small 10422002}
\and
Tran Hai Duong\\
{\tt\small 10422021}
\and
Vo Thi Hong Ha\\
{\tt\small 10421015}
\and
Nguyen Hoang Anh Khoa\\
{\tt\small 10422037}
\and
Truong Hao Nhien\\
{\tt\small 10422062}
\and
Nguyen Song Thien Phuc\\
{\tt\small 10422067}\\
\\
\{\tt @student.vgu.edu.vn\}
\and
Bui Duc Xuan\\
{\tt\small 10422085}
}

\maketitle
%\thispagestyle{empty}



% MAIN ARTICLE GOES BELOW
%%%%%%%%%%%%%%%%%%%%%%%%%%%%%%%%%%%%%%%%%%%%%%%%%%%%%%%%%%%%%%%


%%%%%%%%% ABSTRACT
\begin{abstract}
   In this project, we conducted a comprehensive comparative analysis of prominent machine learning models, namely Alexnet, VGG, Resnet, and YOLO, with a focus on their efficacy in image recognition. Leveraging a curated dataset representative of diverse real-world scenarios with CIFAR-10, our study delved into the nuances of each model's architecture, training process, and computational requirements. Through rigorous evaluation using metrics such as accuracy, precision, and recall, our results reveal nuanced performance distinctions. Notably, Resnet demonstrated superior accuracy, VGG excelled in feature extraction, YOLO showcased real-time efficiency, and Alexnet exhibited a stable performance. These findings provide valuable insights for practitioners and researchers seeking to optimize model selection for specific applications, shedding light on the trade-offs between accuracy, computational cost, and real-time processing capabilities. Project's detailed code are provided at {\url{https://github.com/nhientruong04/LIA-introCS-proj}}.
\end{abstract}

%%%%%%%%% BODY TEXT
\section{Literature reivew}

Deep learning methodologies have been widely applied in the detection and classification of images , with a multitude of research focusing on improving their precision and effectiveness. Each research study is unique, considering factors such as the specific type of tasks, the deep learning methods used, the performance metrics applied, and the datasets chosen. These factors could potentially affect the applicability of the models to different datasets. By categorizing these studies based on the specific type of tasks and the deep learning method used, we can identify similarities and differences, which could provide valuable insights for our current research. A significant number of studies have utilized convolutional neural networks (CNN) for object classification . For example, Krizhevsky et al.\cite{Alexnet} demonstrated the power of deep convolutional neural networks with AlexNet, a CNN with 8 convolutional layers and shows that it achieves a top-1 error rate of 15.3\% on the ImageNet classification task. This success paved the way for the development of deeper CNNs such as VGG and ResNet. Simonyan et al.\cite{VGG}  present a novel architecture for convolutional neural networks (CNNs) that enables the training of extremely deep networks (VGGNets)  and achieved state-of-the-art results,with a top-1 error rate of 5.1\% on the ImageNet classification task. Similarly, Zhang et al.\cite{ResNet} introduced the Deep Residual Learning (DRL) framework, which achieves record-breaking results  attaining a 3.57\% top-1 error rate for 152 layers on the ImageNet classification task. DRL introduces residual connections, which allow for the construction of much deeper networks without vanishing gradients. Despite these promising results , these CNNs models are limited in training and data related. Some studies employ other pipelines which include advanced techniques to get a better result in classification. Redmon et al.\cite{Yolo} used an unified framework algorithm, YOLO. With the assistance of the framework, it performs both object detection and classification in a single pass of the input image. YOLO is able to execute within a short period of time , while achieving comparable accuracy. These four models have been extensively studied and evaluated in the literature. For example, a comprehensive review of deep learning models for object detection by Vaswani et al. (2020)\cite{Review} compared the performance of AlexNet, VGG, ResNet, and YOLO on a variety of object detection datasets. The review found that ResNet and YOLO generally outperformed AlexNet and VGG on all datasets. These models have been shown to achieve state-of-the-art results on a variety of image recognition and object detection tasks. However, AlexNet, VGG, ResNet, and YOLO are still widely used in practice due to their simplicity, robustness, and accuracy.


\section{Insightful summarization}


\begin{table}[h]
\centering
\label{tab:small_table}
\scalebox{0.8}{\begin{tabular}{|c|c|c|c|c|}
\hline

Models & Release Date & Number of layers & Params (M) & Flops(G)\footnote{}\\
\hline
        AlexNet & 2012 & 8 & 61 & 0.715 \\
        VGGNet & 2014 & 16 or 19 & 138-144 & 15.5-20 \\
        ResNet-50 & 2015 & 50 & 25.56 & 4.12 \\
        ResNet101 & 2015 & 101 & 44.55 & 7.85 \\
        Yolov5x-cis & 2020 & 19 & 48.1 & 15.9 \\
        Yolov8x-cls & 2023 & 53 & 57.4 & 154.8 \\
\hline
\end{tabular}}
\caption{Data Comparision}
\end{table}
In addition to the aspects mentioned above, models that were created afterwards fixed the weaknesses of their predecessors. For instance,  AlexNet lacks explicit regularization techniques, making it prone to overfitting. VGGNet incorporates dropout, a regularization technique that randomly drops out a certain percentage of neurons during training. This forces the model to learn more robust and generalizable features, reducing overfitting and improving generalization performance. Although AlexNet implies dropout as well but only in the first two fully connected layers, VGGNet has it in both convolutional and connected layers. However, these two plain networks still confront the degradation problem when it comes to extending their architectures. Hence, Resnet was made to solve the vanishing gradients problem as the layers went deeper and deeper. Yolo is more like a pipeline that includes CNN models as backbone and others applies advanced techniques into training. Leading to a significant increase in FLOPs in terms of the numbers of params. 








\footnotetext{GigaFlop (or Gflop) is a billion FLOPS. Here we take the data of the models that train with ImageNet dataset}
{\small
\bibliographystyle{ieee}
\bibliography{bibliography.bib}
}

\end{document}